%%%%%%%%%%%%%%%%%%%%%%%%%%%%%%%%%%%%%%%%%%%%%%%%%%%%%%%%%
%%
%%					PAPER CONFIG
%%
%%					This configures the paper environment.
%%					The biggest thing is to set the root directory appropriately.
%%%%%%%%%%%%%%%%%%%%%%%%%%%%%%%%%%%%%%%%%%%%%%%%%%%%%%%
%%
\newcommand*{\RootDirectory}{"/Users/Mark/Documents/UW/Research/H5+/paper"}

%%%%%%%%%%%%%%%%%%%%%%%%%%%%%%%%%%%%%%%%%%%%%%%%%%%%%%%%%%%%%%%%%%%%%
%% This is a (brief) model paper using the achemso class
%% The document class accepts keyval options, which should include
%% the target journal and optionally the manuscript type.
%%%%%%%%%%%%%%%%%%%%%%%%%%%%%%%%%%%%%%%%%%%%%%%%%%%%%%%%%%%%%%%%%%%%%
\documentclass[journal=jacsat, manuscript=article, layout=twocolumn]{achemso}

%%%%%%%%%%%%%%%%%%%%%%%%%%%%%%%%%%%%%%%%%%%%%%%%%%%%%%%%%%%%%%%%%%%%%
%% If issues arise when submitting your manuscript, you may want to
%% un-comment the next line.  This provides information on the
%% version of every file you have used.
%%%%%%%%%%%%%%%%%%%%%%%%%%%%%%%%%%%%%%%%%%%%%%%%%%%%%%%%%%%%%%%%%%%%%
% \listfiles

\usepackage{import}

\import{\RootDirectory}{config/packages.tex}

\import{\RootDirectory}{config/custom.tex}

\import{\RootDirectory}{config/author.tex}

\import{\RootDirectory}{config/title.tex}

\import{\RootDirectory}{config/keywords.tex}

%\begin{document}

\hfive{}, having five atoms, has nine vibrational modes. We focus on those modes built from the coordinates shown in \reffig{h5geom}\COM{not quite right, but not worth fixing}. In our model, there are two coordinates describing the position of the shared \hplus{} and two for stretches in the \htwo{}s. We justify this choice of coordinates by considering that prior work in our group and others studying \hfive{} with discrete variable representation~\cite{Lin2012}~\cite{Lin2015} and diffusion Monte Carlo~\cite{Lin2013}~\cite{Lin2015}~\cite{Marlett2015} has found that the most intense transitions could be characterized as excitations of the shared \hplus{} modes and the  \htwo{} modes. Harmonic calculations using the Gaussian 09 program\cite{Gaussian09} also suggest that the bending motions that comprise the five unaccounted for bending vibrations carry little to no oscillator strength.

\loadfig{h5geom}

%\end{document}
